\documentclass{imports}
\usepackage[margin=1in]{geometry}

\title{\textbf{Analysis of Algorithms I}\\ k - Vertex Disjoin Path Problem}
\author{}
\date{}

\begin{document}
\maketitle
\allowdisplaybreaks % allow equations to be spit across files

    The given problem "The mutually avoiding path problem" or the "k - Vertex disjoint paths problem" is NP-Complete.
    We prove it be reducing 3-SAT to it. \cite{kddmain} \cite{stackoverflowkdd} \vspace{10pt}

    Given a solution of $k$ vertex disjoint paths, we simply check if each of the path's vertices and edges do exist in the main graph
    in polynomial time by traversing the adjacency matrix/list of the solution and comparing values with the adjacency matrix/list of the main graph.
    Also, to check the disjoint-ness, we can create an array of booleans of size $|V|$, where $|V|$ is as many vertices in the main graph. Each slot represents each vertex's presence
    in the $k$ disjoint paths. Now, we simply perform the travesal described above again and mark each vertex in the array. If during the traversal
    any one slot of the array already has a truth value present in it, then the paths are not disjoint. If until the end of the traversal that does not
    happen then they are a valid solution. Clearly, this is a polynomial time operation as well. As verifying the solution is a polynomial time
    operation the problem is in NP. \vspace{10pt}

    Let the 3-SAT problem have $m$ clauses and $n$ variables. To overarching idea is to construct paths representing
    a satisfying assignment to each variable and to each clause and relating them in such a way that there should be $m+n$
    independent paths to represent a successful assignment. \vspace{10pt}
    
    For each variable $x_i$ ($1<=i<=n$) let there be a start and end vertex $vs_i, vt_i$. And let there be as many $vT_{ij}, vF_{ij}$
    ($1<=j<=m$) vertices as there are occurences of that variable in the $m$ clauses . We connect $vs_i$ to the first $vT_{ij}$ and connect
    that to the next $vT_{ij}$ and so on until the kast $vT_{ij}$ and then connect that to $vt_i$. We repeat the same and connet $vs_i, vF_{ij}..., vt_i$
    the same way. This way we have 2 paths going out from $vs_i$ and 2 paths converging on $vt_i$. Each of those paths represent a truth assignment
    or a false assignment for that variable. \vspace{10pt}

    Next, for each clause $c_i$ $(1<=i<=m)$ let there be a start and end vertex $cs_i, ct_i$. And let there be as many $l_{ij}$ vertices as 
    there are literals in that clause ($i<=j<=3$). We first connect $cs_i$ to all of the $l_{ij}$. Now if $l_{ij}$ is the unnegated variable
    $x_k$, then we connect it to $vF_{ki}$ (i.e the False vertex for variable $x_k$ for the current clause $i$). If $l_{ij}$ is the negated
    variable $x_k$, then we connect it to $vT_{ki}$ (i.e the Truth vertex for variable $x_K$ for the current clause $i$). Then we connect 
    the $vF_{ki}$ or the $vT_{ki}$ chosen to $ct_i$. We do this for all clauses and each literal variable for that clause.
    \vspace{10pt}

    This conversion of the 3-SAT to a graph problem conveys $k$ vertex disjoint paths problem. There needs to be vertex independent
    paths for each variable and each clause. If we pick, say the truth assignment for variable $x_i$, then the path 
    $vs_i \to vT_{i1} \to  vT_{i2} \dots \to vt_i$ is taken. Which means, all the clauses that have this variable $x_i$
    in the unnegated form still have vertices left to choose from to create their own paths as they were connected to the $vF_{ij}$.
    Similarly, those clauses which have this variable in the negetaed form and are thus connected to the $vT_{ij}$ can no longer form a 
    path as $vT_{ij}$ vertices have already been chosen. This is representative of the clause because if $x_i$ is True, then the clause
    with $x_i$ acheives its truth value from $x_i$ (and thus has a path using $vF_{ij}$), and the clause with $x_{i}'$ cannot
    acheive its truth value from $x_i$ (and thus cannot form a path using $vT_{ij}$). \vspace{10pt}

    This conversion of 3-SAT to the graph takes polynomial time as we add a fixed number of vertices for each clause and variable,
    i.e. polynomial in the size of the input. \vspace{10pt}

    Now, given a $k = m+n$ vertex disjoint paths solution to the converted problem, extracting the truth assignment is polynomial time
    operation too. Visit each vertex $vs_i$ and check which way it branches out. If it branches out to $vT_{ij}$, then variable $x_i$ has
    a truth value. If it branches out to $vF_{ij}$, then it has a false value. This is the 3-SAT satisfying assignment.
    \vspace{10pt}

    To prove that if there is no solution to the k - vertex disjoint paths problem then there is no 3-SAT solution, let us prove the 
    contrapositive. i.e. if there is a 3-SAT assignment, then there is a k vertex disjoint paths solution. Let there be a truth assignment
    such that variable $x_i$ is true. As already discussed above, in the k - vertex disjoint paths problem there will be a path for 
    variable $x_k$ in the form of $vs_k \to vT_{k1} \to  vT_{k2} \dots \to vt_k$. And simultaneouly that path's selection will still qualify
    every other clause $c_i$ that has $x_k$ to have a vertex disjoint path via $cs_i \to vF_{ki} \to ct_i$. This applied for each 
    variable. And so, if a 3-SAT solution exists then the k - vertex disjoint paths problem solution also exists. \vspace{10pt}

    Therefore, the k - vertex disjoint paths problem is NP-Complete. \vspace{10pt}

    With the vertex disjoint constraint this is a very easy problem. All we have to do is apply k DFS searches starting from $a_i$
    to $b_i$. Each DFS operation is polynomial time complexity, so the entire $k$ DFS operations stays polynomial too.





    \vspace{10pt}
    \cite{team}
    \newpage
    \bibliographystyle{unsrt}
    \bibliography{npc}
\end{document}