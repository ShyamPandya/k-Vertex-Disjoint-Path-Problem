\documentclass{imports}
\usepackage[margin=1in]{geometry}

\title{\textbf{Analysis of Algorithms I}\\ k - Vertex Disjoin Path Problem}
\author{}
\date{}

\begin{document}
\maketitle
\allowdisplaybreaks % allow equations to be spit across files

    The given problem "The mutually avoiding path problem" or the "k - Vertex disjoint paths problem" is NP-Complete.
    We prove it by reducing 3-SAT to it. \cite{kddmain} \cite{stackoverflowkdd} \vspace{10pt}
    
    To prove the problem's NP-completeness, we need to first show that it is a search problem in NP.\\
    
    \textbf{$k$ - Vertex disjoint paths problem in NP}\\

    Given a solution of $k$ vertex disjoint paths, we simply check if each of the path's vertices and edges do exist in the main graph
    in polynomial time by traversing the adjacency matrix/list of the solution and comparing values with the adjacency matrix/list of the main graph.
    Also, to check the disjoint-ness, we can create an array of booleans of size $|V|$, where $|V|$ is as many vertices in the main graph. Each slot represents each vertex's presence
    in the $k$ disjoint paths. Now, we simply perform the travesal described above again and mark each vertex in the array. If during the traversal
    any one slot of the array already has a truth value present in it, then the paths are not disjoint. If until the end of the traversal that does not
    happen then they are a valid solution. Clearly, this is a polynomial time operation as well. As verifying the solution is a polynomial time
    operation the problem is in NP. \vspace{10pt}

    Let the 3-SAT problem have $m$ clauses and $n$ variables. The overarching idea is to construct paths representing
    a satisfying assignment to each variable and to each clause and relating them in such a way that there should be $m+n$
    independent paths to represent a successful assignment. \vspace{10pt}

    \textbf{3-SAT problem input reduction in polynomial time}\\
    
    For each variable $x_i$ ($1<=i<=n$) let there be a start and end vertex $vs_i, vt_i$. And let there be as many $vT_{ij}, vF_{ij}$
    ($1<=j<=m$) vertices as there are occurences of that variable in the $m$ clauses . We connect $vs_i$ to the first $vT_{ij}$ and connect
    that to the next $vT_{ij}$ and so on until the last $vT_{ij}$ and then connect that to $vt_i$. We repeat the same and connect the vertices $vs_i, vF_{ij}..., vt_i$. This way we have 2 paths going out from $vs_i$ and 2 paths converging on $vt_i$. Each of those paths represent a truth assignment
    or a false assignment for that variable. \vspace{10pt}

    Next, for each clause $c_i$ $(1<=i<=m)$ let there be a start and end vertex $cs_i, ct_i$. And let there be as many as $l_{ij}$ vertices as 
    there are literals in that clause ($j<=3$). We first connect $cs_i$ to all of the $l_{ij}$. Now if $l_{ij}$ is the unnegated variable
    $x_k$, then we connect it to $vF_{ki}$ (i.e the False vertex for variable $x_k$ for the current clause $i$). If $l_{ij}$ is the negated
    variable $x_k$, then we connect it to $vT_{ki}$ (i.e the Truth vertex for variable $x_K$ for the current clause $i$). Then we connect 
    the $vF_{ki}$ or the $vT_{ki}$ chosen to $ct_i$. We do this for all clauses and each literal variable for that clause.
    \vspace{10pt}.

    This conversion of 3-SAT to the graph takes polynomial time as we add a fixed number of vertices for each clause and variable,
    i.e. polynomial in the size of the input. \vspace{10pt}

    \textbf{$k$ - Vertex disjoint paths problem solution reduction in polynomial time}\\

    Now, given a $k = m+n$ vertex disjoint paths solution to the converted problem, extracting the truth assignment is polynomial time
    operation too. Visit each vertex $vs_i$ and check which way it branches out. If the path starting from the branch of $vT_{ij}$ is part of the solution, then variable $x_i$ has a truth value. If it branches out to $vF_{ij}$, then it has a false value. This is the 3-SAT satisfying assignment.
    \vspace{10pt}

    Now, we need to prove that if there is a solution to the 3-SAT problem then there is a solution to the $k$ - Vertex disjoint paths problem as well and vice versa.\\

    Let there be a truth assignment to the 3-SAT problem with $m$ clauses and $n$ variables. If in the problem, a variable $x_i$ has been assigned $True$, then in the reduced graph that we created it is equivalent to choosing the path $vs_i \to vT_{i1} \to  vT_{i2} \dots \to vt_i$ (All occurrences of $x_i$ have to be $True$). This means, that all the clauses that have this variable $x_i$ in the unnegated form still have the vertices left to choose from to create their own paths as they were connected to the corresponding $vF_{ij}$ vertex. Similarly, clauses which have this variable in the negated form and are thus connected to the $vT_{ij}$ can no longer form a  path as then have already been chosen. This is representative of the clause because if $x_i$ is True, then the clause with $x_i$ acheives its truth value from $x_i$ (and thus has a path using $vF_{ij}$). By using the truth assignment of the remaining variables in the same way, we see that we find $m+n=k$ paths that are vertex disjoint in the reduced graph. Thus the solution for a 3-SAT problem, provides the solution to the reduced $k$ - Vertex disjoint paths problem as well. \vspace{10pt}

    Let there be $k = m+n$ vertex-disjoint paths as a solution in the reduced graph generated. If from a source vertex in one of the paths, the branch with edge to $vT_{ij}$ is present, then that can be translated to setting the value of $x_i$ as $True$ in the 3-SAT problem. Checking any other path in the solution, it will either be for a different variable or a clause and as the path is disjoint, any assignment of a different variable will not interfere with our current variables assignment. As for the path starting from a corresponding clause vertex, it cannot include $vT_{ij}$ and thus includes $vF_{ij}$, which means that the literal is present in the unnegated state in the clause and thus is the right assignment to satisfy the clause. Similarly, we can get the corrseponding truth assignment for all variables based on the remaining disjoing paths in the graph and using the same logic, it will satisfy the 3-SAT formula. \vspace{10pt}
 
    This proves that the solution to $k$ - vertex disjoint paths problem exists iff a solution to the 3-SAT problem exits and proves that it is an NP-complete problem. \vspace{10pt}


    \textbf{Easier version of the problem}

    COMMENT -  NEED TO CHANGE THIS PORTION

    Therefore, the k - vertex disjoint paths problem is NP-Complete. 

    With the vertex disjoint constraint this is a very easy problem. All we have to do is apply k DFS searches starting from $a_i$
    to $b_i$. Each DFS operation is polynomial time complexity, so the entire $k$ DFS operations stays polynomial too.





    \vspace{10pt}
    \cite{team}
    \newpage
    \bibliographystyle{unsrt}
    \bibliography{npc}
\end{document}
